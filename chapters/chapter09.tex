\chapter{Conclusions}\label{ch:conclusions}

% Part I
% - description du pourquoi et du comment des arbres (seuls)
% - description du pourquoi et du comment des forets
% - détails d'implémentation
% => ?

% Part II
% - étude des VIs
% - étude du biais
% Pistes:
% - cas fini
% - cas binaires
% - MDA?
% - derive better estimators than those based on trees

% Part III
% - rp
% => no need to work on all data!
% Pistes:
% - dans quelle situation?
% - pourquoi certains pq marchent bien avec ps ou pf?


% Général:
% call for an understanding of all methods, instead blindly using them
%   - autant pour rf que pour de nouvelles methods (i.e., deep learning)

% http://www.pyimagesearch.com/2014/06/09/get-deep-learning-bandwagon-get-
% perspective/ Because when we sit down and think about a problem, when we
% take the time to not only understand what our feature space “is” and what it
% “implies” in the real-world — then we are acting like machine learning
% scientists. Otherwise, we just a bunch of machine learning engineers,
% blindly performing black box learning and operating a set of R, MATLAB, and
% Python libraries. The takeaway is this: machine learning isn’t a tool. It’s
% a methodology with a rational thought process that is entirely dependent on
% the problem we are trying to solve. We shouldn’t blindly apply algorithms
% and see what sticks. We need to sit down, explore the feature space (both
% empirically and in terms of real-world implications), and then consider our
% best mode of action.
