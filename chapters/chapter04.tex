\chapter{Random Forests}\label{ch:forest}

\begin{remark}{Outline}
In this chapter, we present the well-known family of \textit{Random Forests}
methods. In Section~\ref{sec:4:bias-variance}, we first describe the bias-
variance decomposition of the prediction error and then present, in
Section~\ref{sec:4:bagging}, how the Bagging ensemble method reduces prediction
error by decreasing the variance term of this decomposition. In
Section~\ref{sec:4:ensemble}, we revisit Random Forests ant its variants and
study how randomness introduced into the decision tree induction algorithm can
further reduce the prediction error with respect to Bagging. Finally, the
consistency of forests of randomized trees is explored in
Section~\ref{sec:4:consistency}.
\end{remark}

\section{Bias-variance decomposition}
\label{sec:4:bias-variance}

% décomposition de l'erreur en bias+variance/noise
% geurts, 2006, appendix E => est-ce que ça tient la route en termes de correlation effect???

In section~\ref{sec:2:performance-evaluation}, we defined the generalization
error of a model $\varphi_{\cal L}$ as its expected prediction error
according to some loss function $L$
\begin{equation}
Err(\varphi_{\cal L}) = \mathbb{E}_{X,Y} \{ L(Y, \varphi_{\cal L}(X)) \}.
\end{equation}
Similarly, the expected prediction error of $\varphi_{\cal L}$ at $X=\mathbf{x}$
can be expressed as
\begin{equation}
Err(\varphi_{\cal L}(\mathbf{x})) = \mathbb{E}_{Y|X=\mathbf{x}} \{ L(Y, \varphi_{\cal L}(\mathbf{x})) \}.\label{eqn:4:generalization-error:x}
\end{equation}

In regression, this latter form of the expected prediction error additively
decomposes into bias and variance terms which together constitute a very useful
framework for diagnosing the error of a model. In classification
however, a similar decomposition is more difficult to obtain, yet several
bias and variance frameworks have been proposed to provide a similar analysis.

\subsection{Regression error}

In regression, assuming that $L$ is the squared error loss, the expected
prediction error of a model $\varphi_{\cal L}$ at a given point $X=\mathbf{x}$
can be reexpressed in terms of the Bayes model $\varphi_B$:
\begin{align}
Err(\varphi_{\cal L}(\mathbf{x})) &= \mathbb{E}_{Y|X=\mathbf{x}} \{ (Y - \varphi_{\cal L}(\mathbf{x}))^2 \} \nonumber \\
                                  &= \mathbb{E}_{Y|X=\mathbf{x}} \{ (Y -\varphi_B(\mathbf{x}) + \varphi_B(\mathbf{x}) - \varphi_{\cal L}(\mathbf{x}))^2 \} \nonumber \\
                                  &= \mathbb{E}_{Y|X=\mathbf{x}} \{ (Y -\varphi_B(\mathbf{x}))^2 + (\varphi_B(\mathbf{x}) - \varphi_{\cal L}(\mathbf{x}))^2 \nonumber \\
                                  & \quad\quad\quad\quad\quad+ 2 (Y - \varphi_B(\mathbf{x}))(\varphi_B(\mathbf{x}) - \varphi_{\cal L}(\mathbf{x})) \} \nonumber \\
                                  &= \mathbb{E}_{Y|X=\mathbf{x}} \{ (Y -\varphi_B(\mathbf{x}))^2 \} \nonumber \\
                                  & \quad+ \mathbb{E}_{Y|X=\mathbf{x}} \{ (\varphi_B(\mathbf{x}) - \varphi_{\cal L}(\mathbf{x}))^2 \} \label{eqn:4:decomp1}
\end{align}
since $\mathbb{E}_{Y|X=\mathbf{x}} \{ Y - \varphi_B(\mathbf{x}) \} =
\mathbb{E}_{Y|X=\mathbf{x}} \{ Y \} - \varphi_B(\mathbf{x}) = 0$ by definition
of the Bayes model in regression. In this form, the first term in
the last expression of Equation~\ref{eqn:4:decomp1} corresponds to the (irreducible) residual error
$Err(\varphi_B(\mathbf{x}))$ at  $X=\mathbf{x}$ while the
second term represents the discrepancy of $\varphi_{\cal L}$ from the
Bayes model. The further from the Bayes model, the larger the error.

\subsection{Classification error}

\section{Bagging}
\label{sec:4:bagging}

% Décomposition biais-variance
% Geman, Stuart, Elie Bienenstock, and René Doursat. "Neural networks and the bias/variance dilemma." Neural computation 4.1 (1992): 1-58.

    % > Subsampling with replacement => bagging
    % > Deriver l'erreur et montrer qu'on réduit agit sur la variance, biais reste le même
    %   > effet sur la variance (why bagging works, breiman 1996)
    %   > mais correlation effect (hastie)

    % Aggregation
    % > consensus vote
    % > average probability (Hastie, 283+)
    % > average output values

\section{Ensembles of randomized trees}
\label{sec:4:ensemble}

    % Use the term "random forest" as a global algorithm, as breiman does in RF paper
    % incluant:

    % - Bagging
    % - Random features (Kwok & carter (1990), dietterich 1998 (see rf), breiman)
    % - Random thresholds (geurts)
    % - +other kinds of forests (see state-of-the-art in geurts)
    % => discuter sur un exemple les aspects stat/comput/repres

    % Pq la randomization marche mieux que le bagging seul? (decorrelation)
    % Variance reduction and de-correlation effect due to max_features <= p
    %   aim at low correlation between residuals and low error trees

    % Issues tackled by ensemble
    %   Ensemble methods in ML, Dietterich
    %   - stat
    %   - computational
    %   - representational

    % RF do not overfit (theorem 2.3)

    % OOB estimates

\section{Consistency}
\label{sec:4:consistency}

    % > Mise au point
    % > Preuve de consistence des extra-trees?
