\chapter{Exploring variable importances}\label{ch:applications}

\section{Insights from the decomposition}

% - Along terms => Identify interactions of a given order (strongly, weakly)
% - Along pairs => Identify interactions between variables


\section{Importances in binary decision trees}

% Splitting with binary trees on non-binary variables is like having
% binary variables and adjust the probability of selecting them
% => étudier les propriétés de ce truc là => biais

% understanding bias
% - when k>1,
%       * masking effects
%       * high cardinality variables are more likely to be selected, even when both equally uninformative
% - but, even for k=1, the importance is biased?
%   => because trees are not equiprobable! (partial or total explanation?)
%   => because it is not the same conditionings that are evaluated *** (a branch may comprise several values for b (i.e. B<=b plutot que B=b))
%   => ID3-like are not biased in that sense
% - masking effects again (stronger in RF than in ETs because not all thresholds are evaluted)
%
% reproduire experience de Strobl
% - which one is biased?


\section{Applications}

% Explain Vincent paper
% reproduire experience de microarray dans "Manual on setting up, using, and understanding random forests v3. 1/v4"
