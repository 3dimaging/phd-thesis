\chapter{Applications}\label{ch:applications}

% splitting with binary trees on non-binary variables is like having
% binary variables and adjust the probability of selecting them

% understanding bias
% - when k>1, high cardinality variables are more likely to be selected
% - but, even for k=1, the importance is biased? => there will be more nodes splitting on them (to exhaust all values), for binary trees
% - masking effects (stronger in RF than in ETs)

% decompose variable importances into its terms => detect interaction of a given order
% if r is known (or estimated), can't we built more precise importances from RS for size r?

% reproduire experience de microarray dans "Manual on setting up, using, and understanding random forests v3. 1/v4"
