\chapter{Further insights from importances}\label{ch:applications}

\begin{remark}{Outline}
In this chapter, we build upon results from Chapter~\ref{ch:importances} to
further study variable importances as computed from random forests. In
Section~\ref{sec:7:redundant}, we first examine importances for variables that
are redundant. In Section~\ref{sec:7:bias}, we revisit variable importances in
the context of binary decision trees and ordered variables. In this framework, we
highlight various sources of bias that may concurrently happen when
importances are computed from actual random forests. Finally, we present in
Section~\ref{sec:7:applications} some succesful applications
of variable importance measures.
\end{remark}


\section{Redundant variables}
\label{sec:7:redundant}




\section{Bias in variable importances}
\label{sec:7:bias}

% Splitting with binary trees on non-binary variables is like having
% binary variables and adjust the probability of selecting them
% => étudier les propriétés de ce truc là => biais

% understanding bias
% - when k>1,
%       * masking effects
%       * high cardinality variables are more likely to be selected, even when both equally uninformative
% - but, even for k=1, the importance is biased?
%   => because it is not the same conditionings that are evaluated *** (a branch may comprise several values for b (i.e. B<=b plutot que B=b))
%   => bias may be positive or negative!!
% - masking effects again (stronger in RF than in ETs because not all thresholds are evaluted)
%   c'est commme si on on n'utilisait jamais certaines variables de l'encodage binaire
% - est-ce que la non-equiprobabilité des arbres est une source de biais?
%   => étudier la fréquence des conditionnements pour s'en assurer


% => ID3-like are not biased in that sense
%
% reproduire experience de Strobl
% - which one is biased?


\section{Applications}
\label{sec:7:applications}

% Explain Vincent paper
% reproduire experience de microarray dans "Manual on setting up, using, and understanding random forests v3. 1/v4"
