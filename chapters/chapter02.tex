\chapter{Background}\label{ch:background}

\section{Learning from data}

In the examples introduced in Chapter~\ref{ch:introduction}, the objective
which is sought is to find a systematic way of predicting a phenomenon given a
set of measurements. In machine learning terms, this goal is formulated as the
{\it supervised learning} task of infering a model that predicts the value of
an output variable based on the observed values of the input variables. In
medicine for instance, the goal is to find a decision rule (i.e., a model) for
predicting the condition of a patient (i.e., the output value) given a set of
measurements such as age, sex, blood pressure or history (i.e., the input
values).

To give a more precise formulation, let us arrange the set of measurements in a
pre-assigned order, i.e., take the input values to be $x_1, x_2, ..., x_p$,
where $x_i \in {\cal X}_i$ (for $i = 1, ..., p$) corresponds to the value of
the input variable $X_i$. Together, the input values $(x_1, x_2, ..., x_p)$
form a $p$-dimensional input vector taking its values in ${\cal X}_1 \times ...
\times {\cal X}_p = {\cal X}$, where ${\cal X}$ is defined as the input space.
Similarly, let us define as $y \in {\cal Y}$ the value of the output variable
$Y$, where ${\cal Y}$ is defined as the output space.


% def: learning sample

    % { API = array }
    % - Data => variables (types), notation, etc
    % - ML tasks, focus on supervised learning
    %     - Classification regression


\section{Building estimators}
    % { API = fit, predict }
    % - Describe popular algorithms (trees, knn, linear models)


\section{Estimating performance}
    % { API = score }
    % - train/test accuracy => underfitting/overfitting
    % - cv accuracy
    % - metrics
